\chapter{Graph Isomorphismen}

\begin{definition}[Baum]
    Sei $ T = (V, E) $ ein verbundener, azyklischer Graph.
    Dann ist $ T $ ein Baum.
\end{definition}

\begin{definition}[Gewurzelter Baum]
    Sei $ T = (V, E) $ ein Baum und $ r \in V $.
    Dann ist $ T_r = (V, E, r) $ ein gewurzelter Baum mit der \textit{Wurzel} $ r $.
\end{definition}

\begin{definition}[Geordneter Baum]
    Sei $ T = (V, E, r) $ ein gewurzelter Baum und sie $ \preceq \; \subseteq V \times V $ eine lineare Ordnungsrelation.
    Dann ist $ T = (V, E, r, \preceq) $ ein geordneter Baum mit der \textit{sibbling order} $ \preceq $.
\end{definition}

\begin{definition}[Graph Isomorphismus]
    Seien $ G = (V_G, E_G) $, $ H = (V_H, E_H) $ zwei Graphen.
    Eine bijektive Funktion $ \hat{\chi} : V_G \rightarrow V_H $ ist ein Isomorphismus, falls $ \{ x, y \} \in E_G \Leftrightarrow \{ \hat{\chi}(x), \hat{\chi}(y) \} \in E_H $.

    $ \hat\chi $ ist ein Isomorphismus auf den gewurzelten Bäumen $ G $, $ H $ mit den Wurzeln $ r_G $ und $ r_H $, falls zusätzlich $ r_h = \hat\chi(r_G) $ gilt.

    $ \hat\chi $ ist ein Isomorphismus auf den geordneten Bäumen $ G $, $ H $ mit den sibbling orders $ \preceq_G $, $ \preceq_H $, falls zusätzlich $ x \preceq_G y \Leftrightarrow \hat\chi(x) \preceq_H \hat\chi(y) $.
\end{definition}

\begin{definition}[Codes]
    Sei $ T = (V, E, r, \preceq) $ ein geordneter Baum.
    Definiere den \textit{Code} $ c(T) \in \{ 0, 1 \}^* $ von $ T $ rekursiv wie folgt:
    \begin{enumerate}
        \item Ist $ v \in V $ ein Blatt in $ T $, setze $ c(v) = 01 $.
        \item \label{itm:code-rec}
        Sei $ v \in V $ mit Kindern $ v_1 \preceq \dots \preceq v_n $.
        Dann setze $ c(v) = 0 \cdot c(v_1) \cdot \dots \cdot c(v_n) \cdot 1 $.
        \item Definiere schlussendlich $ c(T) = c(r) $.
    \end{enumerate}

    Für Bäume ohne Ordnung, modifiziere den Rekursionsschritt \ref{itm:code-rec} so, dass die Kinder eines Knotens nicht nach $ \preceq $, sondern lexikographisch geordnet werden.
\end{definition}

\begin{definition}[Excentricity]
    Sei $ G = (V, E) $ ein Graph.
    Definiere $ ex_T(v) $, $ v \in V $ als die Länge des längsten Pfades zu einem beliebigen, anderen Knoten $ v' \in V $.
\end{definition}

\begin{remark}
    In einem balancierten Baum hat die Wurzel, d.h. der zentrale Knoten, die kleinste excentricity.
    Generell haben in Bäumen Blätter die größte excentricity.
\end{remark}

\begin{definition}[Zentrum]
    Definiere das Zentrum $ C(G) $ eines Graphen $ G = (V, E) $ als die Menge an Knoten $ \argmin_{v \in V} ex_G(v) $.
\end{definition}

\begin{lemma}
    Sei $ T = (V, E) $ ein Baum.
    Dann gilt $ |C(T)| \leq 2 $ und, wenn $ C(T) = \{ x, y \} $, dann $ \{ x, y \} \in E $.
\end{lemma}

\begin{proof}
    Konstruiere das Zentrum algorithmisch.
    \begin{enumerate}
        \item Falls $ |V| \leq 2 $, trivial.
        \item Konstruiere $ T' = (V', E')$ als:
        \begin{itemize}
            \item $ V' = \{ x \in V \mid \deg_T(x) > 1 \} $
            \item $ E' = E \cap V' \times V' $
        \end{itemize}
        Es gilt nun, dass $ V' \ne \emptyset $, da $ 2 < |V| $ und $ T $ verbunden.
        Außerdem gilt:
        \begin{itemize}
            \item $ ex_T(v) = ex_{T'}(v) + 1 $
            \item $ C(T) = C(T') $
        \end{itemize}
    \end{enumerate}

    Dieser Algorithmus endet notwendig im leeren Baum.
    Der vorherige Baum hatte entweder einen Knoten und keine Kante, oder zwei Knoten mit einer Kante.
\end{proof}

\begin{definition}[Code für Bäume]
    Über das Zentrum eines Graphen kann nun der Code für allgemeine Bäume definiert werden.

    Sei $ T $ ein Baum.
    \begin{enumerate}
        \item Angenommen $ C(T) = \{ v \} $, dann wähle $ v $ als Wurzel und kodiere $ T $ wie einen gewurzelten Baum.
        \item Angenommen $ C(T) = \{ v_1, v_2 \} $.
        Seien $ T_{v_1} $, $ T_{v_2} $ die Teilbäume von $ T \setminus \{\{ v_1, v_2 \}\}$, die $ v_1 $ bzw. $ v_2 $ enthalten.
        Kodiere diese wie gewurzelte Bäume.

        Wähle $ v_1 $ als Wurzel, falls $ c(T_{v_1}) $ lexikographisch kleiner gleich $ c(T_{v_2}) $, $ v_2 $ sonst und kodiere wie gewurzelten Baum.
    \end{enumerate}
\end{definition}

\begin{definition}[Match/Map]
    Seien $ G_1 = (V_1, E_2) $, $ G_2 = (V_2, E_2) $ zwei gelabelte Graphen mit je einer Knoten-Labeling-Funktion $ \alpha_1 $ bzw. $ \alpha_2 $ und einer Kanten-Labeling-Funktion $ \beta_1 $ bzw. $ \beta_2 $.
    Ein \textit{Match} bzw. eine \textit{Map} $ M \subseteq V_1 \times V_2 $ ist eine Menge, für die gilt:
    \begin{enumerate}
        \item \label{itm:match-bij}
        $ M $ ist bijektiv im eigenen Wertebereich.
        \item Wenn $ (x_1, y_1), (x_2, y_2) \in M $, dann $ \{ x_1, x_2 \} \in E_1 \Leftrightarrow \{ y_1, y_2 \} \in E_2 $.
        \item Wenn $ (x, y) \in M $, dann $ \alpha_1(x) = \alpha_2(y) $.
        \item Wenn $ (x_1, y_1), (x_2, y_2) \in M $, dann $ \beta_1(\{ x_1, x_2 \}) = \beta_2(\{ y_1, y_2 \}) $.
    \end{enumerate}
\end{definition}

\begin{proposition}
    Sei $ M $ ein Match auf $ G_1 $, $ G_2 $.
    Dann gilt nach \ref{itm:match-bij}, dass:
    \begin{align*}
        (x, y), (x', y) \in M & \Rightarrow x = x' \\
        (x, y), (x, y') \in M & \Rightarrow y = y'
    \end{align*}
\end{proposition}

\begin{definition}[VF2 Algorithmus]
    Input: Graphen $ G_1 = (V_1, E_2) $, $ G_2 = (V_2, E_2) $ und Match $ M $, Kandidatenmenge $ P \subseteq V_1 \times V_2 $.

    \begin{enumerate}
        \item Falls $ M $ vollständig auf $ V_1 $ oder $ V_2 $ definiert ist, gib $ M $ aus.
        \item \label{itm:vf2-loop}
        Für jedes Paar $ (x, y) \in P $:
        \begin{enumerate}
            \item Falls $ M \cup \{ (x, y) \} $ \textit{zulässig}, setze $ M' := M \cup \{ (x, y) \} $.
            \item Berechne neue Kandidatenmenge $ P' $.
            \item Rekursiver Aufruf mit $ M' $ und $ P' $.
        \end{enumerate}
    \end{enumerate}

    Initial kann der VF2 Algorithmus mit $ M = \emptyset $ und $ P = V_1 \times V_2 $ aufgerufen werden.
    $ P $ kann im Algorithmus, bspw. vor Schritt \ref{itm:vf2-loop}, bezüglich \textit{geschickter} Bedingungen eingeschränkt werden, z.~B. so, dass nur Paare von Knoten mit gleichen Vertex Labels betrachtet werden o. Ä.
\end{definition}

\begin{remark}
    Wichtigster Punkt des VF2 Algorithmus ist die Definition der \textit{zulässigen} Matchings.
    Diese Bedingungen müssen monoton sein.
\end{remark}

\begin{remark}
    Der VF2 Algorithmus berechnet eine Menge von maximalen, gemeinsamen Teilgraphen, die Laufzeit des Algorithmus ist aber theoretisch exponentiell.
    Praktisch hat sich dieser Algorithmus jedoch auch auf großen Instanzen bewährt.
\end{remark}
