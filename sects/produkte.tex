\chapter{Graph Produkte}

\begin{definition}[Produkt]
    Seien $ G, H $ Graphen.
    Definiere $ \boxop $ Operator:
    \begin{itemize}
        \item $ V(G \boxop H) = V(G) \times V(H) $
        \item $ \{ (x, y), (u, v) \} \in E(G \boxop H) $ gdw. \begin{enumerate}
            \item $ \{ x, u \} \in E(G) $ und $ y = v $ oder
            \item $ \{ y, v \} \in E(H) $ und $ x = u $.
        \end{enumerate}
    \end{itemize}
\end{definition}

\begin{proposition}
    Für $ \boxop $ gilt:
    \begin{itemize}
        \item Ein \textit{neutrales Element} existiert: $ G \boxop K_1 \cong G $
        \item Assoziativität: $ (G \boxop H) \boxop I \cong G \boxop (H \boxop I) $
        \item Kommutativität: $ G \boxop H \cong H \boxop G $
    \end{itemize}
\end{proposition}

\begin{theorem}
    Sei $ G $ ein Graph, dann gibt es eine eindeutige Multi-Menge von Graphen $ G_i \neq K_i $ mit:
    \begin{enumerate}
        \item $ G = \boxop^k_{i = 1} G_i $ und
        \item $ G_i $ ist prim, d.h., wenn $ G_i = H_1 \boxop H_2 $, dann $ H_1 = K_1 $ oder $ H_2 = K_1 $.
    \end{enumerate}
\end{theorem}

\begin{remark}
    Eine Primfaktorzerlegung bzgl. des Graph-Produkts lässt sich in Linearzeit berechnen.
\end{remark}

\begin{definition}[Hamming-Graph]
    $ G $ ist ein \textit{Hamming-Graph}, falls er sich für ein $ n \in \nats $ darstellen lässt als ein Produkt vollständig verbundener $ n $ Graphen:
    \begin{equation*}
        \boxop^k_{i = 1} K_n
    \end{equation*}
\end{definition}

\begin{remark}
    Bspw. sind alle $ n $-dimensionalen Hyperwürfel ein Hamming-Graph auf Basis von $ K_2 $.

    Ein Hamming-Graph lässt einen die Hamming-Distanz von Wörter über einem Alphabet der Größe $ n $ direkt aus der Distanz im Graphen ablesen.
    Jeder Knoten im $ K_n $ Graph repräsentiert einen Buchstaben des Wortes, ein Tupel-Knoten im $ K_n^k $-Hamming-Graphen ein Wort.
\end{remark}

\begin{definition}[Direktes Produkt]
    Seien $ G, H $ Graphen.
    Definiere das \textit{direkte Produkt} $ G \times H $ als:
    \begin{itemize}
        \item $ V(G \times H) = V(G) \times V(H) $ und
        \item $ \{ (x, y), (u, v) \} \in E(G \times H) $ gdw. $ \{ x, u\} \in E(G) \land \{ y, v \} \in E(H) $.
    \end{itemize}
\end{definition}

\begin{definition}[Starkes Produkt]
    Seien $ G, H $ Graphen.
    Definiere das \textit{starke Produkt} $ G \boxtimes H $ als:
    \begin{itemize}
        \item $ V(G \boxtimes H) = V(G) \times V(H) $ und
        \item $ E(G \boxtimes H) = E(G \times H) \cup E(G \boxop H) $.
    \end{itemize}
\end{definition}

\begin{proposition}
    Für alle $ p, q \in \nats $ gilt:
    \begin{equation*}
        K_p \boxtimes K_q \cong K_{(p \cdot q)}
    \end{equation*}

    Somit ist $ K_n $ Prim-Graph der vollständigen Graphen bzgl. des starken Produkts gdw. $ n $ ist eine Primzahl.
\end{proposition}

\begin{definition}[Modulares Produkt]
    Seien $ G, H $ Graphen.
    Definiere das \textit{modulare Produkt} $ G * H $ als:
    \begin{itemize}
        \item $ V(G * H) = V(G) \times V(H) $ und
        \item $ \{ (u, x), (v, y) \} \in E(G * H) $ gdw.
        \begin{enumerate}
            \item $ \{ u, v \} \in E(G) $ und $ \{ x, y \} \in E(H) $ oder
            \item $ \{ u, v \} \notin E(G) $ und $ \{ x, y \} \notin E(H) $, aber $ u \ne v $ und $ x \ne y $.
        \end{enumerate}
    \end{itemize}
\end{definition}

\begin{remark}
    Jeder vollständige Teilgraph von $ G * H $ entspricht einem gemeinsamen induzierten Teilgraph von $ G $ und $ H $.
\end{remark}
