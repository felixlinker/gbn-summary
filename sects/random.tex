\chapter{Zufällige Graphen}

Im folgenden sei $ \mathcal{G} $ die Menge aller Graphen und $ G = (V, E) $ mit $ V = \{ 0, \dots, n - 1 \} $ ein fixierter Graph.

\begin{definition}
    Sei $ V $ eine Menge von Knoten.
    \begin{equation*}
        [V]^2 = \{ \{ v_1, v_2 \} \mid v_1, v_2 \in V \}
    \end{equation*}
\end{definition}

\begin{definition}[Elementares Ereignis]
    Sei die Menge an Knoten $ V $ fixiert und $ G_0 $ ein fixierter Graph mit den Knoten $ V $ und $ m $ Kanten.
    Dann ist $ \{ G_0 \} $ das sog. \textit{elementare Ereignis}.

    Sei $ p \in [0, 1]$ die Wahrscheinlichkeit, eine Kante $ e \in E \subseteq [V]^2 $ zu auszuwählen.
    Dann gilt:
    \begin{equation*}
        P(\{ G_0 \}) = p^m \cdot (1 - p)^{\binom{n}{2} - m}
    \end{equation*}
\end{definition}

Im folgenden gelte $ q = 1 - p $.

\begin{definition}
    Sei $ e \in [V]^2 $.
    Definiere den Wahrscheinlichkeitsraum $ \Omega_e = \{ 0_e, 1_e \} $ mit:
    \begin{itemize}
        \item $ P(\{ 1_e \}) = p $ und
        \item $ P(\{ 0_e \}) = 1 - p = q $.
    \end{itemize}
\end{definition}

\begin{definition}
    $ \mathcal{G} = \mathcal{G}(n, p) $ bezeichne den Produktraum über
    \begin{align*}
        \Omega &= \prod_{e \in [V]^2} \Omega_e \\
        &= \{ 0_0, 1_0 \} \times \{ 0_1, 1_1 \} \times \dots
    \end{align*}

    Interpretiere dabei $ \omega \in \Omega $ als Funktion, d.h. $ \omega : [V]^2 \rightarrow \{ 0, 1 \} $.
    $ \omega $ korrespondiert zu dem Graphen $ G_\omega = (V, \{ e \mid \omega(e) = 1 \}) $.
\end{definition}

\begin{remark}
    Das Ziehen einer Menge von Graphen über $ n $ Knoten mit der Kanten-Wahrscheinlichkeit $ p $ ist ein Ereignis in $ \mathcal{G}(n, p) $.

    Insbesondere für $ e \in [V]^2 $ ist $ A_e = \{ \omega \in \Omega \mid \omega(e) = 1 \} $ ein Ereignis.
\end{remark}

\begin{proposition}
    Für $ e \in [V]^2 $ ist $ A_e $ unabhängig.
\end{proposition}

\begin{proof}
    Es ist offensichtlich, dass $ A_e = \{ 1_e \} \times \prod_{e' \in [V]^2, e' \ne e} \Omega_{e'} $.
    Es gilt:
    \begin{align*}
        P(A_e) &= p \cdot \prod_{e' \in [V]^2, e' \ne e} 1 \\
        &= p
    \end{align*}
\end{proof}

\begin{remark}
    Ähnliches gilt für $ \{ e_1, \dots, e_k \} \subseteq [V]^2 $.
    Dann ist $ P(A_{\{ e_1, \dots, e_k \}}) = P(A_{e_1} \cap \dots \cap A_{e_k}) = p^k $.
\end{remark}

\begin{proposition}
    Sei $ H = (V_H, E_H)$ ein fixierter Graph mit $ |V_H| \leq |V| $ und $ |E_H| \leq |E| $.

    Für die Wahrscheinlichkeit $ P(H \subseteq G) $, dass $ G $ zufällig gezogen wird mit $ H \subseteq G $, gilt: $ P(H \subseteq G) = p^{|E_H|} $.

    Für die Wahrscheinlichkeit $ P(H = G[V_H]) $, dass $ G $ zufällig gezogen wird mit $ H = G[V_H] $, gilt $ P(H = G[V_H]) = p^{|E_H|} \cdot (1 - p)^{\binom{|V_H|}{2} - |E_H|} $.
\end{proposition}

\begin{definition}[Unabhängige Mengen]
    Eine Menge $ M \subseteq V $ ist unabhängig, falls $ \forall v_1, v_2 \in M: \{ v_1, v_2 \} \notin E $.
\end{definition}

\begin{definition}[Independence Number]
    Die independence number $ \alpha(G) $ ist die maximale Anzahl unabhängiger Knoten, d.h. die Kardinalität der größten unabhängigen Menge in $ G $.
\end{definition}

\begin{lemma}
    \label{lem:independent-prob}
    Für alle $ n, k \in \nats $ mit $ 2 \leq k \leq n $ gilt für die Wahrscheinlichkeit $ P(\alpha(G) \geq k) $, dass ein zufälliger Graph $ G $ aus $ \mathcal{G}(n, p) $ mindestens eine Menge von $ k $ unabhängigen Knoten besitzt:
    \begin{equation*}
        P(\alpha(G) \geq k) \leq \underbrace{\binom{n}{k}}_{\text{Anzahl Teilmengen}} \cdot \underbrace{(1 - p)^{\binom{k}{2}}}_{\text{Teilmenge ist unabhängig}}
    \end{equation*}
\end{lemma}

\begin{lemma}
    Mit $ k \leq n $ gilt für die Wahrscheinlichkeit $ P(K_k \subseteq G) $, dass ein zufälliger Graph $ G $ $ K_k $ als Teilgraphen hat, d.h., dass die Anzahl Knoten in der größten Clique von $ G $ $ cli(G) $ größer gleich $ k $ ist:
    \begin{equation*}
        P(K_k \subseteq G) = P(cli(G) \geq k) \leq \binom{n}{k} \cdot p^{\binom{k}{2}}
    \end{equation*}
\end{lemma}

\begin{lemma}
    Für $ k \leq n $ gilt: Falls $ P(\alpha(G) \geq k) + P(cli(G) \geq k) < 1 $, dann gibt es Graphen $ G $ ohne entsprechende Clique \textit{und} unabhängige Menge über $ V $.
\end{lemma}

\begin{definition}
    Sei $ X_k : \mathcal{G}(n, p) \rightarrow \nats $ die Zufallsvariable, die einem Graphen über $ V $ die Anzahl an $ k $-Kreisen zuweist.
\end{definition}

\begin{proposition}
    Sei $ \mathcal{C}_k $ die Menge aller $ k $-Kreise in $ G $ modulo Isomorphie.
    Sei:
    \begin{equation*}
        (n)_k = \prod_{0 \leq i \leq k - 1} (n - i) = n \cdot (n - 1) \cdot \dots \cdot (n - k + 1)
    \end{equation*}
    $ (n)_k $ ist die Anzahl an Möglichkeiten, $ k $ unterschiedliche Knoten in $ V $ zu wählen.
    Des Weiteren gibt es zu jedem $ k $-langen Kreis $ 2k $-viele Isomorphie Kreise.

    Es gilt somit:
    \begin{equation*}
        |\mathcal{C}_k| = \frac{(n)_k}{2k}
    \end{equation*}
\end{proposition}

\begin{proposition}
    Der Erwartungswert von $ X_k $ ist:
    \begin{equation*}
        E(X_k) = \frac{(n)_k}{2k} \cdot p^k
    \end{equation*}
\end{proposition}

\section{Färben zufälliger Graphen}

\begin{proposition}
    Sei $ p \in (0, 1) $ und $ \epsilon > 0 $.
    Dann gilt, für fast jeden Graph $ G \in \mathcal{G}(n, p) $:
    \begin{equation*}
        \chi(G) > \frac{\log(\frac{1}{q})}{2 + \epsilon} \cdot \frac{n}{\log(n)}
    \end{equation*}
\end{proposition}

\begin{proof}
    Für $ 2 \leq k \leq n $ gilt nach Lemma \ref{lem:independent-prob}:
    \begin{align*}
        P(\alpha(G) \geq k) & \leq \binom{n}{k} q^{\binom{k}{2}} \\
        & \leq n^k \cdot q^{\binom{k}{2}} \\
        &= q^{k \cdot \frac{\log(n)}{\log(q)} + \frac{1}{2}k(k - 1)} \\
        &= q^{\frac{k}{2} \cdot (-2 \frac{\log(n)}{\log(q^{-1})} + k - 1)}
    \end{align*}

    Für $ k := 2 + \epsilon \frac{\log(n)}{\log(q^{-1})} $ gilt $ \frac{k}{2} \cdot (-2 \frac{\log(n)}{\log(q^{-1})} + k - 1) $ geht gegen $ \infty $ für $ n \rightarrow \infty $.

    Somit gilt für fast jeden Graphen $ G \in \mathcal{G}(n, p) $: In irgendeiner Färbung von $ G $ gibt es keine $ k $ Knoten derselben Farbe.

    Somit sind mehr als $ \frac{n}{k} = \frac{\log(q^{-1})}{2 + \epsilon} \cdot \frac{n}{\log(n)} $ Farben nötig.
\end{proof}

\section{Eigenschaften fast aller Graphen}

\begin{proposition}
    Für $ p \in (0, 1) $ und einen fixen Graph $ H $ gilt, fast jeder Graph $ G \in \mathcal{G}(n, p) $ enthält einen induzierten Graph $ H $.
\end{proposition}

\begin{proof}
    Sei $ H $ gegeben und $ k = |H| $.
    Wenn $ n = |V| \geq k $ und $ U \subseteq V $ mit $ |U| = k $, dann ist $ H $ isomorph zu $ G[U] $ mit Wahrscheinlichkeit $ r > 0 $.
    $ r $ hängt von $ p $, aber nicht $ n $ ab, da $ U $ fixiert.
    $ G $ besteht aus $ \lfloor \frac{n}{k} \rfloor $ vielen disjunkten Mengen $ U $.

    Die Wahrscheinlichkeit, dass keiner dieser induzierten Graphen $ G[U_i] $ isomorph zu $ H $ ist, ist $ (1 - r)^{\lfloor \frac{n}{k} \rfloor} $.
    Diese Ereignisse sind unabhängig, da $ U_i $ paarweise disjunkt.

    Es gilt:
    \begin{equation*}
        P(H \not \subseteq G[U_i], i \in [1, \lfloor \frac{n}{k} \rfloor]) \leq (1 - r)^{\frac{n}{k}} \xrightarrow[n \to \infty]{} 0
    \end{equation*}
\end{proof}
