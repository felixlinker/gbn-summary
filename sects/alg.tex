\chapter{Algebraische Graphentheorie \& Kreise}

Im folgenden sei $ G = (V = \{ v_1, \dots, v_n \}, E = \{ e_1, \dots, e_m \}) $ ein fixierter, ungerichteter Graph.

\begin{definition}[Inzidenzmatrix]
    Eine Inzidenzmatrix $ I $ ist eine Matrix der Dimension $ n \times m $ mit:
    \begin{multicols}{2}
        Für ungerichtete Graphen:
        \begin{equation*}
            I_{kl} = \begin{cases}
                0, & v_k \notin e_l \\
                1, & v_i \in e_l
            \end{cases}
        \end{equation*}
        \columnbreak

        Für gerichtete Graphen:
        \begin{equation*}
            I_{kl} = \begin{cases}
                1, & e_l = (v_k, v') \\
                0, & v_k \notin e_l \\
                -1, & e_l = (v', v_k)
            \end{cases}
        \end{equation*}
    \end{multicols}

    \textit{(Die Wahl der jeweiligen Konstanten ist dabei Konvention.)}
\end{definition}

\begin{definition}[Inzidenzstruktur]
    Eine Inzidenzstruktur ist ein Tupel $ (p, B, I) $ mit $ I \subseteq p \times B $ und $ p \cap B = \emptyset $.
    \begin{itemize}
        \item $ p $ ist eine Menge von \textit{Punkten}
        \item $ B $ ist eine Menge von \textit{Blöcken}
        \item $ I $ ist eine \textit{Inzidenzmatrix}
    \end{itemize}
\end{definition}

\begin{definition}[Adjanzenzmatrix]
    Eine Adjanzenzmatrix ist eine Matrix $ A \subseteq \{ 0, 1 \}^{n \times m} $ mit
    \begin{equation*}
        A_{kl} = 1 \Leftrightarrow \{ v_k, v_l \} \in E
    \end{equation*}
\end{definition}

\begin{definition}[Gradmatrix/Degreematrix]
    Eine Gradmatrix ist eine Matrix $ D \subseteq \nats^{n \times m} $ mit
    \begin{equation*}
        D_{kl} = \begin{cases}
            deg(v_k), & k = l \\
            0
        \end{cases}
    \end{equation*}
\end{definition}

\begin{definition}[Laplacematrix]
    Seien $ D $ eine Gradmatrix und $ A $ eine Adjanzenzmatrix.
    Die Laplacematrix $ L $ ist definiert als:
    \begin{equation*}
        L = D - A
    \end{equation*}

    Für einen gerichteten Graphen und dessen Inzidenzmatrix $ I $, kann $ L $ alternativ definiert werden als: $ L = I \cdot I^\top $.
\end{definition}

\section{Algebraische Konnektivität}

\begin{definition}[Eigenwertproblem]
    Sei $ M \subseteq \reals^{n \times m} $ und $ E \in Diag(1)^n $ die entsprechende Einheitsmatrix.
    Das \textit{Eigenwertproblem} beschreibt das finden von Skalaren $ \lambda $ und Vektoren $ x $, sodass
    \begin{align*}
        & Ax = \lambda Ex \\
        \equiv \; & (A - \lambda E) x = 0
    \end{align*}

    Die Menge aller Eigenwerte $ \lambda_i $ induziert das \textit{Eigenspektrum}.
\end{definition}

\begin{definition}[Graph Spektrum]
    Das \textit{Spektrum eines Graphen} ist das Spektrum der Laplace-Matrix.
\end{definition}

\begin{definition}[Fiedler Wert/Vektor/Algebraische Konnektivität]
    Der Fiedler Wert ist der zweit-kleinste Eigenwert der Laplacematrix (mehrfach vorkommende Eigenwerte werden mehrfach gezählt).
    Der Fiedler Vektor ist der Eigenvektor des Fiedler Werts.

    Der Fiedler Wert ist die \textbf{algebraische Konnektivität} eines Graphen.
\end{definition}

\begin{proposition}
    Die Anzahl der 0-en im Spektrum der Laplacematrix ist gleich der Anzahl der Zusammenhangskomponenten eines Graphen.
\end{proposition}

\begin{corollary}
    Der Fiedler Wert ist 0 gdw. ein Graph ist nicht verbunden.
\end{corollary}

\begin{observation}
    Sei $ n = |V| $, $ D $ der Durchmesser von $ G $ und sei $ G $ $ k $-verbunden.
    Dann gilt, dass die algebraische Konnektivität $ v $ beschränkt ist durch:
    \begin{equation*}
        \frac{4}{nD} \leq v \leq k
    \end{equation*}
\end{observation}

\begin{observation}
    Der Fiedler Vektor eignet sich zur Partitionierung von Graphen.
    Komponenten dieses Vektors beschreiben die algebraische Verbundenheit des entsprechenden Knotens (es wird der Knoten des jeweiligen Index der Laplace-/Adjanzenzmatrix beschrieben).

    Je kleiner der Wert, desto weniger verbunden ist ein Knoten.
    Zerlegt man $ G $ nach Vorzeichen des Fiedler Vektors in zwei Teilgraphen, erhält man zwei ``optimal'' zusammenhängende Komponenten.
\end{observation}

\subsection{Cauchy Interlace Theorem}

\begin{theorem}[Cauchy Interlace Theorem]
    \label{thm:interlace}
    Sei $ M \subseteq \reals^{n \times n} $ eine symmetrische Matrix, d.h. $ M = M^\top $\footnote{%
        Im eigentlich Theorem ist die Matrix definiert auf den komplexen Zahlen und muss lediglich hermitesch sein, d.h. symmetrisch modulo komplexe Konjugation.
        Dies ist hier jedoch nicht von Relevanz.
    }, mit den Eigenwerten $ \lambda_1 \leq \lambda_2 \leq \dots \leq \lambda_n $.
    Des Weiteren sei $ N \subseteq \reals^{n - 1 \times n - 1} $ eine Teilmatrix von $ M $ mit den Eigenwerten $ \mu_1 \leq \mu_2 \leq \dots \leq \mu_{n - 1} $.

    Dann gilt:
    \begin{equation*}
        \lambda_1 \leq \mu_1 \leq \lambda_2 \leq \mu_2 \leq \dots \leq \mu_{n - 1} \leq \lambda_n
    \end{equation*}
    Dies ist die durch das Theorem beschriebene ``interlacing Struktur''.
\end{theorem}

\begin{observation}
    Das Cauchy Interlace Theorem (\ref{thm:interlace}) beschreibt, welche Auswirkungen das Löschen von Knoten auf die Eigenwerte der Laplacematrix und damit Konnektivität des Graphen hat.
\end{observation}

\section{Konnektivität}

\begin{definition}[Unabhängige Pfade]
    Zwei Pfade $ p = x p_1 \dots p_n y $ und $ q = x q_1 \dots q_m y $ sind unabhängig, falls $ p_i \neq q_j $ für $ 1 \leq i \leq n, 1 \leq j \leq n $.
\end{definition}

\begin{proposition}[Charakterisierung von $ k $-Verbundenheit]
    \label{prp:char-k-connected}
    $ G $ ist $ k $-verbunden, falls:
    \begin{enumerate}
        \item Alle $ k - 1 $ großen, induzierten Subgraphen verbunden sind.
        \item Für alle $ v_1, v_2 \in V $, $ k - 1 $-viele, unabhängige Pfade von $ v_1 $ nach $ v_2 $ existieren.
    \end{enumerate}
\end{proposition}

\begin{proposition}[Charakterisierung von 2-Verbundenheit]
    \label{prp:char-2-connected}
    Die Menge aller 2-verbundenen Graphen ist gleich $ \mathcal{V} $.

    $ \mathcal{V} $ sei induktiv definiert durch:
    Sei $ \mathcal{C} $ die Menge aller Kreise.
    $ \mathcal{V} $ ist die kleinste Menge, für die gilt:
    \begin{enumerate}
        \item $ \mathcal{C} \subseteq \mathcal{V} $
        \item Wenn $ G \in \mathcal{V} $ und $ H \notin \mathcal{C} $ ein Pfad mit Endpunkten (und nur Endpunkten) auf $ G $, dann gilt $ G \cup H \in \mathcal{V} $.
    \end{enumerate}
\end{proposition}

\begin{proof}~\par
    \begin{description}
        \item[``$ \subseteq $''] Sei $ G $ 2-verbunden, aber nicht in $ \mathcal{V} $.
        Da $ G $ 2-verbunden, ex. ein Kreis in $ G $.
        Somit ex. ein maximaler, induzierter, echter Subgraph $ H \subsetneq G $ mit $ H \in \mathcal{V} $.

        Da $ G $ verbunden, ex. Kante $ vw \in E $ mit $ w \in H $ und $ v \in G \setminus H $.
        Da $ G $ 2-verbunden, ex. nach Proposition \ref{prp:char-k-connected} ein Pfad zu einem beliebigen anderen Knoten $ w' \in H $, der nicht über $ w $ verläuft.
        Somit existiert in $ G $ aber ein Pfad von $ w $ über $ v $ nach $ w' $.
        Damit aber nach Voraussetzung der Maximalität von $ H $, $ v \in V(H) $.
        Widerspruch.
        \item[``$ \supseteq $''] Offensichtlich.
    \end{description}
\end{proof}

\begin{remark}
    Ähnlich konstruktive Charakterisierungen sind für $ k $-Verbundenheit mit $ 2 < k $ deutlich schwerer.
\end{remark}

\begin{observation}
    Die induktive Definition aus Proposition \ref{prp:char-2-connected} induziert einen Algorithmus zur Generierung 2-verbundener Graphen und Überprüfung, ob ein Graph 2-verbunden ist.
    \begin{enumerate}
        \item Zur Generierung, starte mit einem beliebigen Kreis und füge Pfade hinzu, die selbst kein Kreis sind.
        \item Zur Überprüfung, lösche Pfade aus dem Graphen, die selbst kein Kreis sind, und schau, ob am Ende ein Kreis entsteht.
    \end{enumerate}

    Kreise sind demnach ``atomare'' 2-verbundene Graphen.
\end{observation}

\section{Kreisbasen}

Im Folgenden werden nur zusammenhängende Graphen mit mindestens zwei Knoten betrachtet.
Dadurch lässt sich ein Graph vollständig durch seine Kantenmenge $ E $ beschreiben.
Werden Teilgraphen betrachtet, müssen diese die gleichen Eigenschaften erfüllen.

\begin{remark}
    Die Ergebnisse dieses Abschnitts können auch auf Graphen, die diese Eigenschaften nicht erfüllen, ausgeweitet werden.
    I.~A. bestehen solche Graphen aus Komponenten von 2-zusammenhängenden Graphen, seinen Blöcken.

    Die Ergebnisse in diesem Abschnitt können dann auf die Blöcke angewendet werden.
\end{remark}

\begin{definition}[Charakteristischer Vektor]
    Sei $ F \subseteq E $.
    Der charakteristische Vektor von $ F $, $ \vec{F} \in \{ 0, 1 \}^m $ ist gegeben durch:
    \begin{equation*}
        \vec{F}_i = \begin{cases}
            0, & e_i \notin F \\
            1, & e_i \in F
        \end{cases}
    \end{equation*}
\end{definition}

Im Folgenden werden Mengen von Kanten und charakteristische Vektoren synonym verwendet.
Des Weiteren bezeichne $ \cdot $ die komponentenweise Multiplikation, $ \odot $ die Matrixmultiplikation und $ \oplus $ den XOR Operator.

\begin{proposition}
    \label{prp:xor-odd-even}
    Sei $ A \subseteq E $.
    Es gilt:
    \begin{equation*}
        \bigoplus_{e \in A} e = \begin{cases}
            0, & |A| \mod 2 = 0 \\
            1, & |A| \mod 2 \ne 0
        \end{cases}
    \end{equation*}
\end{proposition}

\begin{proposition}
    Sei $ H $ eine Inzidenzmatrix und $ C \subseteq E $ ein Kreis.
    Dann gilt:
    \begin{equation*}
        H \odot C = 0^m
    \end{equation*}
    oder äquivalent
    \begin{equation*}
        \forall x \in V: (H \odot C)_x = 0 = \bigoplus_{e \in E} (H_{xe} \cdot C_e)
    \end{equation*}
\end{proposition}

\begin{proof}
    Betrachte $ x \in V $.
    Beweis per Fallunterscheidung.
    \begin{enumerate}
        \item Es gelte: $ x $ liegt auf $ C $.

        Dann gilt $ \bigoplus_{e \in E} H_{xe} \cdot C_e = 0 $, da niemals $ H_{xe} = 1 $ ($ x $ ist inzident zu Kante $ e $) und $ C_e = 1 $ (Kante $ e $ ist im Kreis).

        \item Es gelt: $ x $ liegt auf $ C $.

        Dann gibt es genau zwei Kanten auf $ C $, die zu $ x $ inzident sind.
        Somit gilt:
        \begin{align*}
            \bigoplus_{e \in E} H_{xe} &= \big(\bigoplus_{e \in C} H_{xe} \cdot C_e\big) \oplus \big(\bigoplus_{e \in E \setminus C} H_{xe} \cdot C_e\big) \\
            &= (1 \oplus 1 \oplus 0 \oplus 0 \oplus \dots) \oplus (0 \oplus 0 \oplus \dots) \\
            &= 0 \oplus 0 = 0
        \end{align*}
    \end{enumerate}
\end{proof}

\begin{proposition}[Addition von Kreisen]
    \label{prp:circle-add}
    Für zwei Kreise $ C_1, C_2 \subseteq E $ gilt:
    \begin{equation*}
        \vec{C_1} \oplus \vec{C_2} \simeq C_1 \simdif C_2 = (C_1 \cup C_2) \setminus (C_1 \cap C_2)
    \end{equation*}
\end{proposition}

\begin{proposition}
    \label{prp:circle-mult}
    Sei $ F \subseteq E $.
    Es gilt $ H \odot F = 0^m $ gdw. $ F $ ist ein Kreis.
\end{proposition}

\begin{remark}
    Nach Propositionen \ref{prp:circle-add} und \ref{prp:circle-mult} kann jeder Kreis als Kantendisjunkte Summe kleinerer Kreise dargestellt werden.
    Dies motiviert die Definition eines Vektorraums der Kreise.
\end{remark}

\begin{definition}[Kreisraum]
    Sei $ V $ die Menge aller Kreise in $ G $.
    Der \textit{Kreisraum} $ \mathcal{C}(G) $ von $ G $ ist der Vektorraum $ (V, \oplus) $ ohne Skalarmultiplikation, d.h. wenn $ F_1, F_2 \in V $, dann $ F_1 \oplus F_2 \in V $.

    Die Basis dieses Vektorraums wird \textit{Kreisbasis} genannt.
\end{definition}

\begin{definition}[Basis]
    Die \textit{Basis} eines Vektorraums $ \mathcal{V} = (V, \oplus, \odot) $ ist eine Menge $ B = \{ b_1, \dots, b_n \}$ von Vektoren mit:
    \begin{enumerate}
        \item Jeder Vektor $ v \in V $ lässt sich als Linearkombination von Basisvektoren darstellen:
        \begin{equation*}
            v = \bigoplus_{i = 1}^n \lambda_i \cdot b_i
        \end{equation*}
        Die Basis bildet ein \textit{Erzeugendensystem}.

        \item $ B $ ist linear unabhängig und damit insbesondere minimal.
    \end{enumerate}
\end{definition}

\begin{definition}[Lineare Unabhängigkeit]
    Eine Menge von Vektoren $ V = \{ v_1, \dots, v_n \} $ ist linear unabhängig, falls die Gleichung
    \begin{equation*}
        \bigoplus_{i = 1}^m \lambda_i \cdot v_i = 0
    \end{equation*}
    nur die Lösung $ \lambda_1 = \lambda_2 = \dots = 0 $ hat.
\end{definition}

\begin{definition}[Kantenraum]
    Die Menge der Einheitsvektoren $ E $ aus $ \{ 0, 1\}^m $ bildet die Basis das \textit{Kantenraums} $ \mathcal{E}(G) $, der alle Teilgraphen von $ G $ enthält.
\end{definition}

\begin{proposition}
    Es gilt $ \mathcal{C}(G) $ ist ein Unterraum von $ \mathcal{E}(G) $, d.h. insbesondere $ \dim \mathcal{C}(G) \leq \dim \mathcal{E}(G) = m $.
\end{proposition}

\section{Schnitte}

\begin{definition}[Schnitt]
    Eine Menge $ K \subseteq E $ ist ein Schnitt, falls $ G \setminus K $ nicht zusammenhängend ist.
\end{definition}

\begin{definition}[Elementarer Schnitt]
    Der elementare Schnitt für einen Knoten $ v \in V $ ist definiert als $ K_v = \{ e \in E \mid v \in e \} $.
\end{definition}

\begin{definition}[Cut]
    Sei $ A \subsetneq V $, $ A \ne \emptyset $.
    $ Cut(A) = \{ e \in E \mid e \cap A \ne \emptyset, e \cap (V \setminus A) \ne \emptyset \} $, d.h. die Menge der Kanten, die einen Endpunkt in $ A $ und einen Endpunkt in $ V \setminus A $ haben.
\end{definition}

\begin{proposition}
    Sei $ A \subsetneq V $, $ A \ne \emptyset $.
    Es gelten jeweils:
    \begin{enumerate}
        \item $ Cut(A) = Cut(V \setminus A) $ und
        \item $ Cut(A) = \bigoplus_{v \in A} K_v $.
    \end{enumerate}
\end{proposition}

\begin{proof}
    \begin{enumerate}
        \item Offensichtlich.
        \item Betrachte $ e \in E $ mit $ e \subseteq A $.
        Dann ex. genau zwei elementare Schnitte von Knoten aus $ A $, die $ e $ enthalten.
        Somit $ (\bigoplus_{v \in A} K_v)_e = 0 $.

        Betrachte nun $ e \in E $ mit $ e \cap A = \emptyset $.
        Dann ex. kein elementarer Schnitt, der $ e $ enthält, somit ist die entsprechende Vektor-Komponente auch hier gleich $ 0 $.

        Angenommen $ e \in E $ mit $ e \cap A \ne \emptyset $, aber $ e \not \subseteq A $.
        Dann ex. genau ein elementarer Schnitt von Knoten aus $ A $, der $ e $ enthält.
        Somit ist die entsprechende Vektor-Komponente gleich $ 1 $.
    \end{enumerate}
\end{proof}

\begin{corollary}
    Es gilt:
    \begin{equation*}
        Cut(A) \oplus Cut(V \setminus A) = 0 = \bigoplus_{v \in V} K_v
    \end{equation*}

    Somit ist die Menge aller elementaren Schnitte linear abhängig.
\end{corollary}

\begin{definition}[Schnittraum]
    Sei $ V $ die Menge aller Schnitte.
    Der \textit{Schnittraum} $ \mathcal{K}(G) $ von $ G $ ist der Vektorraum $ (V, \oplus) $ ohne Skalarmultiplikation.
\end{definition}

\begin{proposition}
    Seien $ K, C \subseteq E $ ein Schnitt und ein Kreis.
    Dann gilt
    \begin{equation*}
        \bigoplus_{e \in E} K_e \cdot C_e = \bigoplus_{e \in K \cap C} 1 = 0
    \end{equation*}
\end{proposition}

\begin{proof}
    Es genügt, zu zeigen, dass $ |K \cap C| \mod 2 = 0 $ (vgl. Prop. \ref{prp:xor-odd-even}).
    Da $ K $ ein Schnitt ist, muss $ C \setminus K $ eine Menge von Pfaden sein.
    Eine Menge von Pfaden könnte aber nicht durch eine ungerade Anzahl von Kanten zu einem Kreis verbunden werden, somit muss $ |K \cap C| $ gerade sein.
\end{proof}

\begin{proposition}
    Es gilt:
    \begin{equation*}
        \dim \mathcal{K}(G) + \dim \mathcal{C}(G) = |E| = \dim \mathcal{E}(G)
    \end{equation*}
\end{proposition}

\begin{remark}
    Schnitte und Kreise sind also orthogonal.
\end{remark}

\begin{definition}[Spannbaum]
    Der \textit{Spannbaum} eines zusammenhängen Graphen ist der Teilgraph $ W \subseteq G $, der ein zusammenhängender Baum ist und jeden Knoten von $ G $ enthält.
\end{definition}

\begin{remark}
    Ein Baum mit mit $ |V| $ Knoten, enthält $ |V| - 1 $ Kanten.
\end{remark}

\begin{definition}[Erweiterter Cut]
    Sei $ T $ ein Spannbaum von $ G $ und $ e $ eine Kante in $ T $.
    Dann ist $ Cut(T, e) $ die Menge aller Kanten in $ G $, die die Zusammenhangskomponenten verbinden, die sich durch entfernen von $ e $ in $ T $ ergeben.
\end{definition}

\begin{proposition}
    \label{prp:cuts-independent}
    Für einen fixierten Spannbaum $ T $ ist die Menge aller Schnitte $ Cut(T, e) $, $ e \in E(T) $ linear unabhängig.
\end{proposition}

\begin{proof}
    Seien $ T = (V_T, E_T) $, $ e, e' \in E_T $.
    Es gilt, dass $ e \in Cut(T, e') $ gdw. $ e = e' $.
    Somit gilt insbesondere für alle Folgen von Knoten $ e_1, e_2, \dots \in E $ mit $ e_i \ne e $: $ Cut(T, e_1)_e = 0 $, $ Cut(T, e_2)_e = 0 $, etc., und damit $ e \notin Cut(T, e_1) \oplus Cut(T, e_2) \oplus \dots $.
\end{proof}

\begin{proposition}
    Jeder Schnitt kann als eine Summe aus Schnitten $ \{ Cut(T, e) \mid e \in T \} $ für einen fixierten Spannbaum $ T $ dargestellt werden.
\end{proposition}

\begin{proposition}
    \label{prp:dim-bases}
    Beachte, dass $ |\{ Cut(T, e) \mid e \in T \}| = |V| - 1 $.
    Somit
    \begin{equation*}
        \dim \mathcal{K}(G) = |V| - 1
    \end{equation*}
    und daher
    \begin{equation*}
        \dim \mathcal{C}(G) = |E| - \dim \mathcal{K}(G) = |E| - |V| + 1
    \end{equation*}
\end{proposition}

\begin{definition}[Cyc]
    Sei $ T = (V_T, E_T) $ ein Spannbaum und $ e \notin E_T $.
    $ Cyc(T, e) $ sei der eindeutig bestimmte Kreis in $ (V_T, E_T \cup \{ e \}) $.
\end{definition}

\begin{proposition}
    \label{prp:cyc-independent}
    Sei $ T = (V_T, E_T) $ ein Spannbaum.
    Die Menge $ \{ Cyc(T, e) \mid e \in E \setminus E_T \} $ ist linear unabhängig.
\end{proposition}

\begin{proof}
    Dieser Beweis kann analog zum Beweis von Proposition \ref{prp:cuts-independent} geführt werden.
    Dazu genügt es, sich zu verdeutlichen, dass für $ e, e' \in E \setminus E_T $ gilt: $ e \in Cyc(T, e') $ gdw. $ e = e' $.
\end{proof}

\begin{proposition}
    Sei $ T = (V_T, E_T) $ ein Spannbaum.
    Die Menge $ C = \{ Cyc(T, e) \mid e \in E \setminus E_T \} $ ist eine Basis für den Kreisraum $ \mathcal{C}(G) $.
\end{proposition}

\begin{proof}
    Es wurde bereits gezeigt, dass $ C $ linear unabhängig ist (vgl. Prop. \ref{prp:cyc-independent}).
    Es bleibt zu zeigen dass $ |V| = \dim \mathcal{C}(G) $.
    \begin{align*}
        |V| &= |\{ Cyc(T, e) \mid e \in E \setminus E_T \}| \\
            &= |E| - |E_T| \\
            &= |E| - (|V| - 1) \\
            &= |E| - |V| + 1 \\
            &= \dim \mathcal{C}(G) & \text{Prop. \ref{prp:dim-bases}}
    \end{align*}
\end{proof}

\begin{remark}
    Für einen Spannbaum $ T = (V_T, E_T) $ wird die Menge $ \{ Cyc(T, e) \mid e \in E \setminus E_T \} $ auf \textit{Kirchhoff-Basis} genannt.
\end{remark}

\section{Ohren-Zerlegung}

Aus der Charakterisierung von 2-verbundenen Graphen in Proposition \ref{prp:char-2-connected} motiviert die Zerlegung von 2-verbunden Graphen in \textit{Ohren}.

\begin{definition}[Ohr]
    Eine Menge von Kanten bzw. ein Graph $ P = e_1e_2\dots e_m $ ist ein Ohr, falls $ E $ einen Kreis darstellt oder einen Pfad mit $ e_1 = \{ x_0, x_1 \} $, $ e_2 = \{ x_1, x_2 \} $, etc., bildet, für den gilt:
    \begin{itemize}
        \item für $ 1 \leq i \leq m - 1 $ gilt $ \deg(x_i) = 2 $;
        \item und $ 3 \leq \deg(x_0), \deg(x_m) $.
    \end{itemize}
\end{definition}

\begin{theorem}[Whitney-Theorem]
    Ist $ G $ 2-zusammenhängend, dann gibt es ein Ohr $ P $ in $ G $, sodass $ G \setminus P $ wieder 2-zusammenhängend ist.
\end{theorem}

\begin{proposition}
    Sei $ G $ 2-zusammenhängend und $ P $ ein Ohr in $ G $.
    Dann ex. ein Kreis $ C $ in $ G $ mit $ P \subseteq C $.
\end{proposition}

\begin{definition}[Ohrenzerlegung]
    Sei $ G $ ein 2-zusammenhängender Graph.
    Eine Ohrenzerlegung ist eine Folge $ (G_k, P_k, G_{k - 1}, P_{k - 1}, \dots G_0) $ mit $ G = G_k $, $ P_i $ ist ein Ohr, $ G_0 $ ist ein Kreis und $ G_{i - 1} = G_i \setminus P_i $.
\end{definition}

\begin{proposition}
    Jeder 2-zusammenhängende Graph hat eine Ohrenzerlegung.
\end{proposition}

\begin{definition}[Ohrenbasis]
    Sei $ (G_k, P_k, \dots, P_1, G_0) $ eine Ohrenzerlegung von $ G $.
    Definiere $ C_k $ als die Ergänzung von $ P_k $ zu einem Kreis in $ G_k $.

    Dann ist
    \begin{equation*}
        B = \{ C_i \mid 1 \leq i \leq k \} \cup \{ G_0 \}
    \end{equation*}
    die \textit{Ohrenbasis} von $ G $.
\end{definition}

\begin{theorem}
    Eine Ohrenbasis $ B $ von $ G $ ist eine Basis des Vektorraums $ \mathcal{C}(G) $.
\end{theorem}

\begin{proof}
    \begin{enumerate}
        \item Zeige, dass $ B $ linear unabhängig ist.

        Betrachte dazu $ C_i \in B $.
        Es ex. ein eindeutiges Ohr $ P_i \subseteq C_i $.
        Nach Konstruktion gilt, dass $ P_i \not \subseteq C_j $ für $ j \ne i $.
        Somit kann $ C_i $ nicht aus einer Folge von anderen Elementen der Basis erzeugt werden.

        \item Zeige, dass $ |B| = \dim \mathcal{C}(G) $.
        Es gelte als Konvention $ G_i = (V_i, E_i) $ und $ B_i $ bezeichne die Kreisbasis von $ G_i $.

        Zunächst gilt offensichtlich, dass die Anzahl Knoten in $ G_{i + 1} $ gleich Summe der Anzahl innerer Knoten von $ P_i $ und der Anzahl von Knoten in $ G_i $ ist.
        Die Anzahl innerer Knoten in $ P_i $ ist wiederum gleich $ |P_i| - 1 $.

        Ebenso gilt, dass $ |G_{i + 1}| = |E_{i}| + |P_{i}| $.

        Zu zeigen ist, dass $ |B| = |B_k| = k + 1 = \dim \mathcal{C}(G) $.
        Beweis per Induktion über $ i $.
        \begin{description}
            \item[IA] Sei $ P_0 = G_0 $.
            Da $ G_0 $ ein Kreis ist, $ B_0 = \{ G_0 \} $ offensichtlich eine Kreisbasis von $ G_0 $ und es gilt $ |B_0| = 0 + 1 $.
            \item[IV] Es gelte $ |B_i| = i + 1 $.
            \item[IS] Zu zeigen ist $ |B_{i + 1}| = i + 2 $.
            Es gilt:
            \begin{align*}
                |B_{i + 1}| &= |E_{i + 1}| - |V_{i + 1}| + 1 & \text{Prop. \ref{prp:dim-bases}} \\
                &= |G_i| + |P_i| - |V_{i + 1}| + 1 \\
                &= |G_i| + |P_i| - (|V_i| + |P_i| - 1) + 1 \\
                &= |G_i| + |P_i| - |V_i| - |P_i| + 2 \\
                &= |G_i| - |V_i| + 2 \\
                &= |B_i| + 1 & \text{Prop. \ref{prp:dim-bases}} \\
                &= i + 1 + 1 = i + 2 & \text{IV}
            \end{align*}
        \end{description}
    \end{enumerate}
\end{proof}

\section{Weitere Eigenschaften}

\begin{definition}[Fundamental]
    Eine Menge von Kreisen $ \mathcal{F} $ heißt \textit{fundamental}, falls die Kreise in $ \mathcal{F} $ in eine Folge $ (C_0, C_1, \dots, C_k) $ so geordnet werden können, dass für $ 2 \leq i \leq k $ gilt:
    \begin{equation*}
        C_i \setminus \big(\bigcup^{i-1}_{j=1} C_j \big) \ne \emptyset
    \end{equation*}
\end{definition}

\begin{remark}
    In einer fundamentalen Menge enthält jeder Kreis (bis auf einen\footnote{%
        Der Kreis, dessen Kanten alle in anderen Kreisen vorkommen, bildet das erste Folgenglied.
    }) eine Kante, die in keinem anderen Kreis vorkommt.
\end{remark}

\begin{proposition}
    Jede Ohrenbasis ist fundamental.
\end{proposition}

\begin{lemma}
    Eine Menge von Kreisen $ \mathcal{F} $ ist nicht fundamental, wenn es keine Kante gibt, die in genau einem Kreis vorkommt.
\end{lemma}

\begin{definition}[Strikt Fundamental]
    Eine Menge von Kreisen $ \mathcal{F} $ ist \textit{strikt fundamental}, wenn die Eigenschaft der Fundamentalität für jede Ordnung der Kreise gilt.
\end{definition}

\begin{proposition}
    Eine Kirchhoff-Basis ist strikt fundamental, d.h. ist $ B $ eine strikt fundamentale Kreisbasis von $ \mathcal{C}(G) $, dann ex. ein Spannbaum $ T $, sodass $ B $ die Kirchhoff-Basis für $ T $ ist.
\end{proposition}

\begin{definition}[Robust]
    Eine Menge von Kreisen $ \mathcal{F} $ ist \textit{robust}, falls die Kreise in $ \mathcal{F} $ in eine Folge $ (C_0, C_1, \dots, C_k) $ so geordnet werden können, dass für alle $ 2 \leq n \leq k $ gilt:

    \begin{equation*}
        \bigoplus^{k-1}_{i=1} C_i
    \end{equation*}
    ist ein elementarer Kreis.
\end{definition}

\begin{definition}[Matroid]
    Sei $ X $ eine Grundmenge, $ I \subseteq 2^X $.
    $ I $ ist ein \textit{Matroid}, falls:
    \begin{enumerate}
        \item $ \emptyset \in I $,
        \item wenn $ A \in I $ und $ B \subseteq A $, dann $ B \in I $ und,
        \item wenn $ A, B \in I $ und $ |B| < |A| $, dann ex. $ x \in A \setminus B $ mit $ B \cup \{ x \} \in I $.
    \end{enumerate}

    Des Weiteren ist $ A \in I $ \textit{maximal unabhängig}, falls $ \forall x \in X \setminus A: A \cup \{ x \} \notin I $.
\end{definition}

\begin{remark}
    Die Menge der Kreise in einem Graphen $ G $ bildet als Vektorraum auch ein Matroid.
\end{remark}

\begin{definition}[Kanonischer Greedy-Algorithmus]
    Sei $ I $ ein Mengensystem über einer Grundmenge $ X $ und $ w : X \rightarrow \reals^+ $ eine Gewichtungsfunktion.

    Ein \textit{Optimierungsproblem} ist die Suche nach einer unabhängigen Menge $ A \in I $, sodass $ \sum_{x \in A} w(x) $ minimal ist.

    Der kanonische Greedy-Algorithmus ist ein Ansatz für dieses Problem und funktioniert wie folgt:
    \begin{enumerate}
        \item Setze $ B := \emptyset $.
        \item Sortiere $ X $ aufsteigend.
        \item \label{itm:greedy-loop} Sei $ x $ das minimale, also das erste, Element aus $ X $.
        \item Lösche $ x $ aus $ X $.
        \item Falls, $ B \cup \{ x \} \in I $, setze $ B := B \cup \{ x \} $.
        \item Falls $ X \ne \emptyset $, gehe zu \ref{itm:greedy-loop}.
    \end{enumerate}
\end{definition}

\begin{proposition}
    Der kanonische Algorithmus berechnet das korrekte Ergebnis auf einem Mengensystem $ I $ gdw. $ I $ ein Matroid ist.
\end{proposition}

\begin{definition}[Vektor Matroid]
    Sei $ R = (V, \oplus) $ ein Vektorraum ohne Skalarmultiplikation.
    Sei $ E \subseteq V $ beliebig und $ I \subseteq 2^E $ die Menge aller linear unabhängigen Teilmengen von $ E $.
    Dann ist $ I $ ein Matroid und wird \textit{Vektormatroid} genannt.
\end{definition}

\begin{proposition}
    Sei $ I $ ein Vektormatroid.
    Dann bildet jede maximal unabhängige Menge eine Basis des Untervektorraums, der von $ E $ induziert wird.
\end{proposition}

\section{Minimale Kreisbasis}

\begin{definition}[Minimale Kreisbasis]
    Eine Basis von $ \mathcal{C}(G) $ ist eine \textit{minimale Kreisbasis (MCB)}, falls sie $ \leq $-minimal bzgl. $ \sum_{C \in B} |C| $ ist.
\end{definition}

\begin{definition}[Relevant]
    Ein Kreis $ C $ ist \textit{relevant}, falls $ C $ nicht als $ \oplus $-Summe von echt kürzeren Kreisen dargestellt werden kann.

\end{definition}

\begin{proposition}
    Ein Kreis ist in einer MCB enthalten gdw. er ist relevant.
\end{proposition}

\begin{proposition}
    Seien $ C_1, C_2 $ zwei Kreise.
    Dann gilt: wenn $ |C_1| + |C_2| = |C_1 \oplus C_2| $, dann ist $ C_1 \oplus C_2 $ nicht relevant.
\end{proposition}

\begin{proposition}
    Ist ein Kreis relevant, dann ist er auch elementar.
\end{proposition}

\begin{definition}[Essentiell]
    Ein Kreis $ C $ ist essentiell, falls er in jeder MCB enthalten ist.
\end{definition}

\begin{proposition}
    Existiert zu einer Kante $ e $ ein eindeutiger Kreis minimaler Länge ist, der $ e $ enthält, dann ist dieser Kreis essentiell.
\end{proposition}

\begin{definition}[Chord-less]
    Ein Kreis $ C $ heißt \textit{chord-less}, falls er keine zwei Knoten $ v_1, v_2 $ enthält, mit $ \{ v_1, v_2 \} \in E \setminus C $.
\end{definition}

\begin{proposition}
    Ist ein Kreis relevant, dann ist er auch chord-less.
\end{proposition}

\begin{definition}[Isometrie]
    Ein Kreis $ C $ ist isometrisch, falls er zwei Knoten $ v_1, v_2 $ enthält, sodass ein kürzester Weg $ P^* $ zwischen $ v_1 $ und $ v_2 $ existiert, sodass $ P^* \subseteq C $.
\end{definition}

\begin{proposition}
    Ist ein Kreis relevant, dann ist er auch isometrisch.
\end{proposition}

\begin{lemma}
    Sei $ M $ eine MCB.
    $ M_i(M) $ bezeichne die Anzahl der Kreise der Länge $ i $ in $ M $.
    Es gilt offensichtlich $ \sum^{|V|}_{i = 3} M_i(M) \cdot i = \sum_{C \in M} |C| $.

    Darüber hinaus gilt, dass $ M_i(M) = M_i(M') $ für eine zweite MCB $ M' $ mit $ M \ne M' $.
\end{lemma}

\begin{remark}
    Nicht nur die Anzahl der Kanten, die insgesamt in der MCB sind, sondern auch die Anzahl der Kreise pro Länge ist somit fixiert.
\end{remark}

\begin{proof}
    Betrachte zwei MCBs $ M = (C_1, C_2, \dots, C_k) $ und $ M' = (C_1', C_2', \dots, C_k') $ als Folgen von nach Länge aufsteigend sortierten Kreisen.
    Angenommen, es ex. o.B.d.A. ein Index $ i $ mit $ |C_i'|<|C_i| $.
    Wähle $ i $ minimal.

    Seien:
    \begin{itemize}
        \item $ B = { C_1, C_2, \dots, C_{i - 1} } $
        \item $ A = { C_1', C_2', \dots, C_i' } $
    \end{itemize}

    Betrachte das Vektormatroid $ I $ zum Kreisvektorraum $ \mathcal{C}(G) $.
    Es gilt $ A, B \in I $, da beide Mengen jeweils linear unabhängig.
    Da $ I $ ein Matroid ist und $ |B| < |A| $ somit auch $ B \cup \{ C_i' \} \in I $
\end{proof}

\begin{definition}[Kanten-kurz]
    Sei $ C $ ein Kreis in $ G $.
    Sei $ e = \{ v_1, v_2 \} \in E $.
    $ v_1', v_2' \in V $ liegen $ e $ \textit{gegenüber}, falls $ |C| $ ungerade und beide Pfade $ v_1 \dots v_1' $ bzw. $ v_2 \dots v_2' $ in $ C $ ohne $ e $ und $ \{ v_1', v_2' \} $ gleiche Länge haben.

    Ansonsten liegt $ e $ dem Knoten $ x \in V $ gegenüber, falls die Pfade $ v_1 \dots x $ und $ v_2 \dots x $ in $ C $ ohne $ e $ haben gleicher Länge sind.

    $ C $ heißt \textit{kanten-kurz}, falls für jede Kante $ e \in E $ gilt, dass es keinen Pfad in $ G $ zu den gegenüberliegenden Knoten $ v_1', v_2' \in V $ gibt, der kürzer ist als die jeweils kürzesten Pfade von $ v_1 $ und $ v_2 $ aus.
    (Angemerkt sei, dass ggfs. $ v_1' = v_2' $, falls $ |C| $ ungerade.)
\end{definition}

\begin{proposition}
    Ist ein Kreis relevant, dann ist er auch kanten-kurz.
\end{proposition}
