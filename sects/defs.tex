\chapter{Grundlegende Definitionen}

\begin{definition}[Graph]
    Ein Graph $ G = (V, E) $ ein Tupel einer endlichen Knotenmenge $ V $ und einer endlichen Kantenmenge $ E \subseteq \{ \{ v, w \} \mid v, w \in V \} $.
\end{definition}

\begin{definition}[Digraph/Gerichteter Graph]
    Ein \textit{Digraph} ist ein Tupel $ G = (V, E) $ mit einer endlichen Knotenmenge $ V $ und einer endlichen Kantenmenge $ E \subseteq V \times V $.
\end{definition}

\begin{remark}
    Es gibt folgende alternative Graphen-Typen:
    \begin{itemize}
        \item Graphen mit Schleifen bzw. Loops
        \item Graphen mit Mehrfachkanten
        \begin{itemize}
            \item Anstelle von einer Menge sind Kanten über eine \textit{tail}-Funktion $ t : E \to V $ und eine \textit{sink}-Funktion $ h : E \to V $ definiert.
        \end{itemize}
        \item Hypergraphen (vgl. Kapitel \ref{chpt:hypergraphs})
        \item Multihypergraphen, d.h. Hypergraphen mit Mehrfachkanten
    \end{itemize}
\end{remark}

Im folgenden Sei $ G = (V, E) $ ein fixierter Graph.
In dieser VL werden (außer im Konkreten angemerkt) nur ungerichtete, schleifenfreie Graphen betrachtet.

\begin{definition}
    Seien $ V_1 \subseteq V $, $ E_1 \subseteq E $ und $ G' = (V', E') $ ein Graph.
    Es seien:
    \begin{itemize}
        \item $ G[V_1] = (V_1, E \cap V_1 \times V_1) $ der von $ V_1 $ \textit{induzierte Teilgraph} von $ G $,
        \item $ G \setminus V_1 = G[V \setminus V_1] $ und
        \item $ G \setminus E_1 = (V, E \setminus E_1) $.
        \item $ G \setminus G' = (G \setminus V') \setminus E' $.
        \item $ G \cup G' = (V \cup V', E \cup E') $.
        \item $ G \cap G' = (V \cap V', E \cap E') $.
    \end{itemize}
\end{definition}

\begin{definition}[Kantenzug/Walk]
    Ein \textit{Kantenzug} der Länge $ l $ ist eine Folge $ (x_0, e_1, x_1, e_2, \dots, e_{l - 1}, x_l) $.
    Ein Kantenzug ist:
    \begin{itemize}
        \item \textit{geschlossen} bzw. ein \textit{Kreis}, falls $ x_0 = x_l $,
        \item ein \textit{Weg}, falls $ e_i \ne e_j $ mit $ i \ne j $,
        \item ein \textit{Pfad}, falls $ x_i \ne x_j $ mit $ i \ne j $ und
        \item ein \textit{elementarer Kreis}, falls er geschlossen und $ (x_1, e_2, \dots, e_{l - 1}, x_l) $ ein Pfad ist.
    \end{itemize}
\end{definition}

\begin{lemma}
    Seien $ x, y \in V $ mit $ x \ne y $.
    Dann gilt:
    \begin{enumerate}
        \item Ex. Kantenzug von $ x $ nach $ y $ gdw.
        \item ex. Weg von $ x $ nach $ y $ gdw.
        \item ex. Pfad von $ x $ nach $ y $.
    \end{enumerate}

    Gilt zusätzlich $ x \ne y $, dann ex. auch ein Kreis, der $ x $ enthält.
\end{lemma}

\begin{definition}[Grad]
    Der \textit{Grad} $ \deg(v) $ eines Knoten $ v \in V $ ist die Anzahl der mit ihm in Verbindung stehender Knoten, d.h. $ \deg(v) = |\{ w \mid \{ v, w \} \in E \}| $.

    Der \textit{maximale Grad} $ \Delta(G) $ eines Graphen ist $ \max_{v \in V} \deg(v) $.
    Der \textit{minimale Grad} $ \delta(G) $ eines Graphen ist $ \min_{v \in V} \deg(v) $.
\end{definition}

\begin{definition}[Abstand]
    Der \textit{Abstand} zwischen zwei Knoten $ v_1, v_2 \in V $ ist die Länge des kürzesten Pfades zwischen diesen Knoten.
    Existiert kein solcher Pfad, gilt $ d(v_1, v_2) = \infty $.
\end{definition}

\begin{observation}
    $ d : V \times V \rightarrow \nats $ ist eine Metrik:
    \begin{enumerate}
        \item $ d(v_1, v_2) = 0 $
        \item $ d(v_1, v_2) = d(v_2, v_1) $
        \item $ d(v_1, v_3) \leq d(v_1, v_2) + d(v_2, v_3) $
        \item $ 0 \leq d(v_1, v_2) $
    \end{enumerate}
\end{observation}

\begin{definition}[Euler'scher Graph]
    $ G $ ist ein \textit{Euler'scher Graph}, falls ein geschlossener Weg existiert, der alle Knoten abdeckt.
\end{definition}

\begin{proposition}
    $ G $ ist ein Euler'scher Weg gdw. $ G $ ist zusammenhängend und jeder Knoten hat geraden Grad.
\end{proposition}

\begin{definition}[Algorithmus Euler'sche Graphen]
    Zur Konstruktion eines entsprechendes Wegs, ein sog. \textit{Eulerkreis}, in einem Euler'schen Graphen kann dieser Algorithmus verwendet werden:
    \begin{enumerate}
        \item Wähle beliebigen Knoten $ v_0 $ und konstruiere einen Kreis $ K $, der keine Kante zwei mal durchläuft.
        \item \label{itm:euler-rec}
        Ist $ K $ ein Eulerkreis, brich ab.
        \item Lösche $ K $ aus $ G $.
        \item Suche ersten Knoten $ v_0^+ $ in $ K $, dessen Grad größer als $ 0 $ ist und suche einen Unterkreis $ K' $, der keine Kante zwei mal durchläuft.
        \item Füge $ K' $ in $ K $ an Stelle des Knoten $ v_0^+ $ ein und gehe zu \ref{itm:euler-rec}.
    \end{enumerate}
\end{definition}

\begin{definition}[Durchmesser]
    Der \textit{Durchmesser} $ D $ von $ G $ ist definiert als:
    \begin{equation*}
        D = \max_{v_1, v_2 \in V} d(v_1, v_2)
    \end{equation*}

    D.h., der Durchmesser ist die Länge des längsten kürzesten Wegs in einem Graphen.
\end{definition}

\begin{definition}[Verbundenheit]
    $ G $ ist verbunden, falls für alle Knoten $ v_1, v_2 \in V $ gilt, dass $ d(v_1, v_2) < \infty $, d.h. zwischen allen Knoten existiert ein Pfad.
\end{definition}

\begin{definition}[Cut vertex]
    $ v \in V $ ist ein cut vertex, falls $ G \setminus \{ v \} $ mehr Zusammenhangskomponenten als $ G $ hat.
\end{definition}

\begin{definition}[Cutset]
    $ V_1 \subsetneq V $, falls $ G \setminus V_1 $ mehr Zusammenhangskomponenten als $ G $ hat.
\end{definition}

\begin{definition}[$ k $-(vertex-)Verbundenheit]
    $ G $ ist $ k $-(vertex-)verbunden, falls eine Knotenmenge $ K \subseteq V $ mit $ |K| = k $ existiert, sodass $ G \setminus K $ nicht mehr verbunden ist und $ G $ nicht $ k' $-verbunden für $ k' < k $ ist, d.h. jedes cutset hat mindestens $ k $ Elemente.
\end{definition}

\begin{proposition}
    $ G $ ist 2-verbunden gdw. $ G $ ist verbunden und enthält keinen cut vertex.
\end{proposition}

\begin{definition}[Block]
    $ B = (V_B, E_B) \subseteq G $ ist ein Block, falls $ B $ maximal 2-verbunden ist.
\end{definition}

\begin{definition}[Blockgraph]
    $ G_B $ von $ G $, sodass ein Knoten in $ G_B $ einen Block in $ G $ repräsentiert.
    Zwei Blöcke sind in $ G_B $ miteinander verbunden, falls eine Kante zwischen zwei Knoten der Blöcke in $ G $ existiert.
\end{definition}

\begin{proposition}
    Jeder Blockgraph ist ein Baum bzw. Wald.
\end{proposition}

\begin{definition}[End Block]
    Ein Knoten $ v \in V_B $ in einem Blockgraphen $ G_B = (V_B, E_B) $ ist ein end block, falls $ \deg(v) \leq 1 $, d.h. $ v $ ist ein Blatt im Blockgraph.
\end{definition}
