\chapter{Grundlegende Definitionen}

Im folgenden Sei $ G = (V, E) $ ein fixierter Graph.

\begin{definition}[Abstand]
    Der \textit{Abstand} zwischen zwei Knoten $ v_1, v_2 \in V $ ist die Länge des kürzesten Pfades zwischen diesen Knoten.
    Existiert kein solcher Pfad, gilt $ d(v_1, v_2) = \infty $.
\end{definition}

\begin{observation}
    $ d : V \times V \rightarrow \nats $ ist eine Metrik:
    \begin{enumerate}
        \item $ d(v_1, v_2) = 0 $
        \item $ d(v_1, v_2) = d(v_2, v_1) $
        \item $ d(v_1, v_3) \leq d(v_1, v_2) + d(v_2, v_3) $
        \item $ 0 \leq d(v_1, v_2) $
    \end{enumerate}
\end{observation}

\begin{definition}[Durchmesser]
    Der \textit{Durchmesser} $ D $ von $ G $ ist definiert als:
    \begin{equation*}
        D = \max_{v_1, v_2 \in V} d(v_1, v_2)
    \end{equation*}

    D.h., der Durchmesser ist die Länge des längsten kürzesten Wegs in einem Graphen.
\end{definition}

\begin{definition}[Verbundenheit]
    $ G $ ist verbunden, falls für alle Knoten $ v_1, v_2 \in V $ gilt, dass $ d(v_1, v_2) < \infty $, d.h. zwischen allen Knoten existiert ein Pfad.
\end{definition}

\begin{definition}[$ k $-(vertex-)Verbundenheit]
    $ G $ ist $ k $-(vertex-)verbunden, falls eine Knotenmenge $ K \subseteq V $ mit $ |K| = k $ existiert, sodass $ G|_{V \setminus K} $ nicht mehr verbunden ist und $ G $ nicht $ k' $-verbunden für $ k' < k $ ist.
\end{definition}
