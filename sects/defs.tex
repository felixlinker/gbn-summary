\chapter{Grundlegende Definitionen}

\begin{definition}[Graph]
    Ein Graph $ G = (V, E) $ ein Tupel einer endlichen Knotenmenge $ V $ und einer endlichen Kantenmenge $ E \subseteq \{ \{ v, w \} \mid v, w \in V \} $.
\end{definition}

Im folgenden Sei $ G = (V, E) $ ein fixierter Graph.

\begin{definition}
    Seien $ V_1 \subseteq V $, $ E_1 \subseteq E $ und $ G' = (V', E') $ ein Graph.
    Es seien:
    \begin{itemize}
        \item $ G[V_1] = (V_1, E \cap V_1 \times V_1) $ der von $ V_1 $ \textit{induzierte Teilgraph} von $ G $,
        \item $ G \setminus V_1 = G[V \setminus V_1] $ und
        \item $ G \setminus E_1 = (V, E \setminus E_1) $.
        \item $ G \setminus G' = (G \setminus V') \setminus E' $.
        \item $ G \cup G' = (V \cup V', E \cup E') $.
        \item $ G \cap G' = (V \cap V', E \cap E') $.
    \end{itemize}
\end{definition}

\begin{definition}[Abstand]
    Der \textit{Abstand} zwischen zwei Knoten $ v_1, v_2 \in V $ ist die Länge des kürzesten Pfades zwischen diesen Knoten.
    Existiert kein solcher Pfad, gilt $ d(v_1, v_2) = \infty $.
\end{definition}

\begin{observation}
    $ d : V \times V \rightarrow \nats $ ist eine Metrik:
    \begin{enumerate}
        \item $ d(v_1, v_2) = 0 $
        \item $ d(v_1, v_2) = d(v_2, v_1) $
        \item $ d(v_1, v_3) \leq d(v_1, v_2) + d(v_2, v_3) $
        \item $ 0 \leq d(v_1, v_2) $
    \end{enumerate}
\end{observation}

\begin{definition}[Durchmesser]
    Der \textit{Durchmesser} $ D $ von $ G $ ist definiert als:
    \begin{equation*}
        D = \max_{v_1, v_2 \in V} d(v_1, v_2)
    \end{equation*}

    D.h., der Durchmesser ist die Länge des längsten kürzesten Wegs in einem Graphen.
\end{definition}

\begin{definition}[Verbundenheit]
    $ G $ ist verbunden, falls für alle Knoten $ v_1, v_2 \in V $ gilt, dass $ d(v_1, v_2) < \infty $, d.h. zwischen allen Knoten existiert ein Pfad.
\end{definition}

\begin{definition}[Cut vertex]
    $ v \in V $ ist ein cut vertex, falls $ G \setminus \{ v \} $ mehr Zusammenhangskomponenten als $ G $ hat.
\end{definition}

\begin{definition}[$ k $-(vertex-)Verbundenheit]
    $ G $ ist $ k $-(vertex-)verbunden, falls eine Knotenmenge $ K \subseteq V $ mit $ |K| = k $ existiert, sodass $ G \setminus K $ nicht mehr verbunden ist und $ G $ nicht $ k' $-verbunden für $ k' < k $ ist.
\end{definition}

\begin{proposition}
    $ G $ ist 2-verbunden gdw. $ G $ ist verbunden und enthält keinen cut vertex.
\end{proposition}

\begin{definition}[Block]
    $ B = (V_B, E_B) \subseteq G $ ist ein Block, falls $ B $ maximal 2-verbunden ist.
\end{definition}
