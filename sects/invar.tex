\chapter{Graph-Invarianten}

\begin{definition}[Graph-Invariante]
    Sei $ \mathcal{G} $ die Menge aller Graphen.
    Eine Graph-Invariante ist eine Funktion $ \psi: \mathcal G \rightarrow U $ mit beliebigem Bildbereich $ U $, falls $ G_1 \cong G_2 \Rightarrow \psi(G_1) = \psi(G_2) $.
\end{definition}

\begin{remark}
    Je nach Güte der Matches kann die Distanz über Matches eine Metrik darstellen.
\end{remark}

\begin{example}
    Beispiele für Graph-Invarianten:
    \begin{itemize}
        \item Anzahl Knoten,
        \item Anzahl Kanten,
        \item maximaler/minimaler/durchschnittlicher Grad,
        \item Summe des Grads,
        \item maximale/minimale/durchschnittliche Pfadlänge,
        \item Anzahl Kreise,
        \item Länge des kürzesten Kreises,
        \item Größe der minimalen Kreisbasis,
        \item Anzahl Zusammenhangskomponenten,
        \item Anzahl Spannbäume,
        \item Algebraische Konnektivität,
        \item Spektral-Invarianten,
        \item etc.
    \end{itemize}
\end{example}

\begin{definition}[Vollständig]
    Eine Graph-Invariante $ \psi $ heißt \textit{vollständig}, falls $ \psi(G_1) = \psi(G_2) \Rightarrow G_1 \cong G_2 $.
\end{definition}

\begin{definition}[Wiener Index]
    Sei $ d $ die Distanz-Matrix eines Graphen $ G $.
    Dann ist der Wiener-Index $ W $ von $ G $ definiert als:
    \begin{equation*}
        W = \frac{1}{2} \sum_{x, y \in V} d_{xy}
    \end{equation*}
\end{definition}

\begin{proposition}
    Der Wiener-Index ist eine Graph-Invariante.
\end{proposition}

\begin{definition}[Degree-Sequence]
    Sei $ G = (V, E) $ ein Graph.
    Die \textit{Degree-Sequence} von $ G $ ist die Folge $ (d_0, d_1, \dots, d_{|V| - 1}) $ wobei $ d_i = |\{ v \in V \mid \deg(v) = i \}| $, also die Anzahl der Knoten mit Grad $ i $.
\end{definition}

\begin{proposition}
    Die degree-sequence ist eine Graph-Invariante.
\end{proposition}

\begin{proposition}
    Sei $ (d_0, \dots, d_n) $ eine die Degree-Sequence von $ G $.
    Dann gilt:
    \begin{itemize}
        \item $ (\sum_i d_i) \mod 2 = 0 $
        \item $ (\sum_i d_i) \leq \binom{|V|}{2} \cdot 2 $
        \item Ex. $ i $ mit $ d_i \geq 2 $, falls $ G $ zusammenhängend.
    \end{itemize}
\end{proposition}

\begin{example}
    Invarianten sind besonders relevant im Bereich der Quantitative Structure Activity Relationsships (QSAR).
    Im Gebiet der QSAR wird versucht, chemische Eigenschaften von Verbindungen vorherzusagen.

    Versuche bestimmtes Aktivitätsmaß $ \alpha(G) $ über Trainingsdaten vorherzusagen.
    Das zu lernende Modell ist dabei:
    \begin{equation*}
        \alpha(G) = \sum c_k \cdot \psi_k(G)
    \end{equation*}
    für eine Menge von Graph-Invarianten $ \Psi = \{ \psi_1, \dots, \psi_n \} $.
\end{example}

\begin{example}
    Eine weitere Verwendung von Graph-Invarianten sind Datenbanken.
    Um schnell überprüfen zu können, ob ein Graph in einer Datenbank ist, können diese gestaffelt nach mehreren Invarianten abgelegt werden.
    Graphen der Datenbank, bei denen alle Invarianten mit denen des Eingabegraphen übereinstimmen, können dann auf Isomorphie getestet werden.
\end{example}

\begin{definition}[Isospektral Paar]
    Zwei Graphen $ G_1 $, $ G_2 $ bilden ein \textit{isospektral-Paar}, falls das Spektrum bzgl. einer Matrix (bspw. Adjazenz, Laplace) von $ G_1 $ und $ G_2 $ identisch ist.
\end{definition}

\begin{remark}
    Praktisch gilt: Grad-basierte Invarianten und Spektral-Invarianten verhalten sich orthogonal, d.h. sie decken unterschiedliche Eigenschaften ab und lassen sich daher gut kombinieren.
\end{remark}
